\documentclass{article}
\input{preamble}

\begin{document}

\section{Outcomes}
\label{sec:outcomes}

 Here's a short summary of what tangible output this study might create. First,
 there are a few main themes,
\begin{itemize}
  \item Visual description of RNN functions. The operation from time $t$ to $t +
    1$ is just a function (albeit a high-dimensional one), but I almost never
    see people visualize this. To deal with dimensionality, we might need some
    reduction, selection of interesting features, or interactivity, but it
    should be reasonably straightforwards.
  \item Empirical exploration of gating behavior. The warping time paper is
    beautiful, but I would like observe some of this warping behavior more
    directly.
  \item Equivalence classes of gating architectures. The authors of warping time
    suggest that a primary difference between different RNN architectures is how
    they warp time. Can we gather some data about how architectures relate, on
    this dimension?
\end{itemize}

In terms of tangible output, we will have

\begin{itemize}
  \item A text summary of the main ideas in the warping RNN.
  \item Visual summaries of the mechanics of RNNs, in low-dimensional toy
    examples, as well as a higher-dimensional real-world example (e.g.,
    character modeling on war and peace). The visual summaries should be
    thought-through interactive visualizations, probably referring to the
    state-of-the-art in the visualization literature for high-dimensional
    time-series visualization.
  \item A low-dimensional simulation experiment to make warping as obvious as it
    can ever be. The purpose is to provide a sanity check of the warping paper
    in a setting that no one can really debate. It doesn't answer the question
    of how relevant the analysis is to real-world sequence data, but it's an
    important initial check.
  \item A secondary version of the low-dimensional experiment, repeated over $>
    2$ of common gating architectures. This element starts getting at point (3)
    above, but in a setting that is still straightforwards.
  \item An interactive visualization of the low-dimensional simulation
    experiment. There are many series and many gating functions that we would
    like access, and we should think through the best way to navigate them.
  \item An experiment extracting gating values on real data. I'm thinking
    character modeling on War and Peace. Since the warping behavior is not known
    apriori, we need to look the types of consistent patterns that emerge. So,
    this has two types of tangible output: a static cluster / ordination plot
    relating types of input sentences in terms of their warping behavior, and an
    interactive display that gives access to the observed warping behaviors.
    Bonus points if you can find a way for users to input their own sentences,
    and see the gating behavior emerge in real-time.
\end{itemize}

I'm aiming for an interactive visualization venue, like Distill, MLVis, or
Observables.

\section{Outline}
\label{sec:outline}

Here's a tentative outline,

\begin{itemize}
\item Introduction
  \begin{itemize}
    \item I'd like to share some views of the time-warping behavior of RNNs.
      There are already great visualizations that can help you understand RNNs
      more generally, but I'd like to focus on visualizing RNNs and their
      relation to invariance.
  \end{itemize}
\item The Cast
  \begin{itemize}
  \item Datasets
    \begin{itemize}
    \item Warped sinusoids. Exactly what it sounds like. Sinusoids with known
      linear or quadratic time warping.
    \item War and peace: The book, as in K's blog post / ICLR paper back in the
      day.
    \end{itemize}
  \item Algorithms
    \begin{itemize}
    \item The basic RNN
    \item LSTM and GRU variants
    \item Some form of interactive visualization, probably on simple dataset,
      using either PCA or linking, probably.
    \item Can mention that an important, but invisible, part of the cast are the
      visualization strategies. They're like the setting for the novel, not ever
      directly addressed, but crucial to the narrative.
    \end{itemize}
  \end{itemize}
\item Conflict: A new take on gating
  \begin{itemize}
  \item Old story: Long term memory, prevents gradients from exploding.
  \item New kids on the block: It's all about invariance. Cool, cause it's like
    relativity.
  \item Who is right?? **dramatic music**
  \end{itemize}
\item Development: A look into simple gating
  \begin{itemize}
  \item Revisiting the sinusoid data. Look at the learned gatings.
  \item Can even look at the learned gatings over time.
  \item Can even assess the concordance between known warping functions and
    learned gatings. This is one dramatic reveal.
  \end{itemize}
\item Conflict: Is any of this real?
  \begin{itemize}
  \item Either way, the previous analysis is inconclusive. Who actually knows
    whether warping is relevant in the oh-so-many applications where RNNs are
    the go-to analysis method?
  \end{itemize}
\item Development: Look into real-world gating
  \begin{itemize}
  \item K. would think we're silly for still looking into war and peace 10 years
    later, but let's have a go at it.
  \item What are gatings like for different sentences?
  \item How do gatings differ across levels of abstraction?
  \end{itemize}
\item Conflict: What's the deal with all these architectures?
  \begin{itemize}
  \item It's nice that we learned something, but let's be real, no one does
    anything like this in the real world (you of course use the new DeepMind
    HydraRNN with Omniscient Attention Modules).
  \item To what extent can we compare some of the different methods used in
    practice, from the point of view of time warping?
  \end{itemize}
\item Development: An empirical comparison
  \begin{itemize}
  \item Let's just look at the time warping behavior of different architectures.
  \item How rich are the function classes learned by different warpings? How
    consistent are they with one another?
  \end{itemize}
\item Denoument
  \begin{itemize}
  \item What have we learned?
  \item Can these be used as diagnostics? Can these used to improve or unify
    architectures?
  \item Can these be applied to other settings (Graphs, RL, ...)?
  \end{itemize}
\end{itemize}

\end{document}
